\documentclass[../main.tex]{subfiles}

\begin{document}

\chapter*{Висновки}
\addcontentsline{toc}{chapter}{\MakeUppercase{Висновки}}
	
Результатом виконання дипломної роботи є розроблений Android-щоденник для мандрівників з планувальником та можливістю запису треку переміщень. Цей додаток дозволить корисувачам зберігати враження віж подорожей, записувати свої переміщення піж час мандрівок, записувати плани та встановлювати нагадування. Додаток також дає можливість синхронізувати дані між різними пристроями для відповідного акаунту. Завдяки поєднанню цих можливостей можна швидно та зручно зберігати моменти поїздок.
	
Для того, щоб створити власний Android-щоденник було розглянуто існуючі додатки-аналоги та проведено аналіз їх переваг та недоліків.  На основі проведеного аналізу, було сформовано вимоги до Android-щоденника, які мали наступний вигляд:

\begin{enumerate}
	\item Авторизація за допомогою облікового запису Google.
	\item Можливість створення подорожі.
	\item Можливість створення записів щоденника.
	\item Можливість створення записів планувальника.
	\item Можливість встановлення нагадувань.
	\item Можливість запису треку переміщень.
\end{enumerate}

Після визначення вимог, що необхідні для створення програмного продукту, було проведено архітектурне проектування, сформовано діаграму пакетів та компонентів майбутнього додатку. Під час детального проектування створено діаграму моделей, на якій зображено моделі та зв'язки між ними. Після етапу проектування було створено діаграму розгортання на апаратних засобах та визначено системні вимоги до додатку.

Для розробки Android-щоденника було обрано мову програмування Java, а в якості середовища розробки -- Android Studio IDE, оскільки Java має чудову підтримку в цьому середовищі. Також, Android Studio є офіційним та найкращим середовищем для розробки Android-додатків.

Потім було сформовано логічну структуру додатку та розроблено інтерфейс користувача, використовуючи правила матеріального дизайну. Після цих етапів розроблений Anroid-щоденник було успішно протестовано. 

Результатом даної дипломної роботи є Android-щоденник для мандрівників з GPS-трекером та планувальником. Отриманий додаток пройшов через усі етапи проектування програмного забезпечення -- від розгляду аналогів до реалізації та тестування. Отриманий продукт у майбутньому буде перетворено у більш складну систему. Він отримає додаткові функції та реалізації для інших платформ, таких як web чи iOS.

Поставлені до Android-щоденника завдання було виконано, а сформовані вимоги~--~задоволено тому він повністю готовий до використання.

	
\end{document}

