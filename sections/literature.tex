\documentclass[../main.tex]{subfiles}

\begin{document}

\chapter*{Список використаних джерел}
\addcontentsline{toc}{chapter}{\MakeUppercase{Список використаних джерел}}
	
\begin{enumerate}
	\item Android. Adapter [Електронний ресурс]~–~Режим доступу: https://developer.android.com/reference/android/widget/Adapter.html (дата звернення 10.05.2017) – Назва з екрана.
	\item Android Studio. The Official IDE for Android [Електронний ресурс]~–~Режим доступу: https://developer.android.com/studio/index.html (дата звернення 10.05.2017) – Назва з екрана.
	\item Barb Darrow. Firebase secures its real-time back-end service [Електронний ресурс]~–~Режим доступу: https://gigaom.com/2012/12/18/firebase-secures-its-real-time-back-end-service (дата звернення 10.05.2017) – Назва з екрана.
	\item Firebase. Getting started [Електронний ресурс]~–~Режим доступу: https://firebase.google.com (дата звернення 10.05.2017) – Назва з екрана.
	\item Google Maps APIs. Android [Електронний ресурс]~–~Режим доступу: https://developers.google.com/maps/android (дата звернення 10.05.2017)~–~Назва з екрана.
	\item Material Design. Guidelines [Електронний ресурс]~–~Режим доступу: https://material.io/guidelines (дата звернення 10.05.2017) – Назва з екрана.
	\item GPS [Електронний ресурс]~–~Режим доступу: https://uk.wikipedia.org/wiki/GPS (дата звернення 10.05.2017) – Назва з екрана.
	\item Об'єктно-орієнтоване програмування [Електронний ресурс]~–~Режим доступу: https://uk.wikipedia.org/wiki/Об'єктно-орієнтоване\_програмування (дата звернення 10.05.2017) – Назва з екрана.
%	\item Object-oriented programming [Електронний ресурс]~–~Режим доступу: https://en.wikipedia.org/wiki/Object-oriented\_programming
	\item Подорож [Електронний ресурс]~–~Режим доступу: https://uk.wikipedia.org/wiki/Подорож (дата звернення 10.05.2017)~–~Назва з екрана.
	\item Башар~А. Groovy и Grails. Практические советы~/~Абдул-Джавад~Башар.~–~М.:~ДМК~Пресс, 2010.~–~392~с.
	\item Авраменко~В.С. Методичні рекомендації щодо виконання дипломних робіт для студентів денної та заочної форм навчання напряму підготовки 6.050103 «Програмна інженерія»~/~В.С.~Авраменко, С.В.~Голуб, В.І.~Салапатов.~–~Черкаси:~ЧНУ ім.~Богдана Хмельницького, 2015.~–~100~с.
	\item Авраменко~В.С. Проектування інформаційних систем~/~В.С.~Авраменко, С.В.~Голуб, В.І.~Салапатов. Електронне видання.~–~Черкаси:~ЧНУ ім.~Богдана Хмельницького, 2015.~–~496~с. Режим доступу: http://www.fotius.cdu.edu.ua
	\item Сиерра~К. Изучаем Java~/~К.~Сиерра, Б.~Бейтс.~–~М.:~Эксмо, 2016.~–~720~с.
	\item Криспин~Л. Гибкое тестирование: практическое руководство для тестировщиков ПО~/~Лайза~Криспин.~–~М.:~Вильямс, 2010.~–~464~с.
	\item Майер~Р. Android 4. Программирование приложений для планшетных компьютеров и смартфонов~/~Рето~Майер.~–~М.:~Эксмо, 2013.~–~816~с.
	\item Фаулер~М. NoSQL: новая методология разработки нереляционных баз данных~/~М.~Фаулер, П.~Садаладж.~–~М.:~Вильямс, 2013.~–~192~с.
	\item Дейтел~П. Android для разработчиков~/~П.~Дейтел, Х.~Дейтел, А.~Уолд.~–~СПб.:~Питер, 2016.~–~512~с.
	\item Kotlin~–~конкурент Java и Scala [Електронний ресурс]~–~Режим доступу: https://www.osp.ru/os/2011/07/13010422 (дата звернення 10.05.2017)~–~Назва з екрана.
	\item Хостманн~К. Scala для нетерпеливых~/~Кей~Хостманн.~–~М.:~ДМК~Пресс, 2015.~–~408~с.
	\item Эккель~Б. Философия Java~/~Брюс~Эккель.~–~СПб.:~Питер, 2016.~–~1168~с.
\end{enumerate}
	
\end{document}

