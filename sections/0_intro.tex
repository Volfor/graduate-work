\documentclass[../main.tex]{subfiles}

\begin{document}

\chapter*{Вступ}

Мобільний телефон давно став для людини предметом однієї з найперших потреб. В наш час більшість людей не уявляє свого життя без можливості щодня здійснювати дзвінки, відправляти повідомлення чи користуватися інтернетом. В столітті високих технологій існує безліч засобів, які спрощують життя та допомагають заощадити час, що для сучасної людини є важливим чинником. Більшість людей відчувають себе не комфортно без наявності телефону. Люди, які ведуть активний спосіб життя, мають потребу в відслідковуванні свого серцевого ритму і втрачених калорій. Так, ті, хто подорожують, мають потребу в збереженні вражень чи плануванні поїздок. Для задоволення таких потреб люди активно використовують мобільні додатки.

Подорожі завжди були і є актуальними. Вони дозволяють людині вийти за межу повсякденності, змінюють її буденне сприйняття і свідомість. У подорожі людина відчуває себе та приймає світ не так як зазвичай. Мандрівка дає людині відчуття свободи. Це відчуття є потужним психологічним фактором, що робить мислення і емоції більш розкутими. Саме в подорожах людині приходять в голову цікаві та нестандартні думки, а отримані враження і нові картини світу пробуджують уяву та тягу до творчості. Тому, багато людей ведуть свій щоденник чи блог, записуючи яскраві враження від поїздок. 

Важливо пам'ятати всі яскраві моменти подорожей, тому дуже добре мати під рукою додаток, який би допоміг у цьому. Більшість Android-щоденників дозволяють користувачам тільки створювати записи та додавати фото. Деякі містять планувальник чи карту. Але, здебільшого доводиться користуватися декількома додатками для задоволення всіх потреб.

Метою роботи є створення мобільного додатку для операційної системи Android, який буде містити щоденник, нагадування (планувальник) та GPS-трекер.  
При виконанні роботи потрібно виконати такі задачі:
\begin{itemize}
\item Дослідити особливості додатків для мандрівників
\item Проаналізувати вже існуючі додатки за схожим жанром
\item Оглянути засоби реалізації подібних програмних продуктів
\item Сформулювати постановку задачі для розробки Android-щоденника
\item Сформувати та проаналізувати вимоги до додатку
\item Провести проектування програми
\item Розробити код програмного продукту
\item Провести тестування розробленого додатку
\end{itemize}

Об'єктом дослідження даної роботи є методи та засоби розробки мобільних додатків. 

Предметом дослідження є методи та засоби розробки Android-додатків для мандрівників

Практична цінність створюваного програмного продукту полягає в тому, що за допомогою нього користувачі можуть зберігати враження від подорожей, додавати фото, записувати маршрут по якому вони рухаються, записувати свої плани та встановлювати нагадування. Також, синхронізація між пристроями дозволить користувачам, які мають аккаунт, легко отримати доступ до своїх даних з інших пристроїв. Наявність всіх цих можливостей в одному додатку дає змогу швидко та зручно зберігати моменти поїздок.

% Дипломна робота складається із вступу, трьох розділів, висновків, списку використаної літератури (_ найменувань), _ додатків. Робота містить _ таблиць, _  рисунків. Роботу викладено на _ сторінках друкованого тексту. 

\end{document}